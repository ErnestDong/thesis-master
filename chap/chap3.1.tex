%!root = ../main.tex
\chapter{进一步的分析}\label{chap:4.1}
在基础回归的基础上,本文将在基准回归的基础上进行进一步的分析,主要探讨灾区平均保额在极端降水冲击后降低的可能原因。

在分析灾区平均保额在极端天气冲击后降低的现象时,本文提出冲击导致损失不足、低估冲击再次发生概率和新低保额保单摊薄三个可能的解释:

首先,降水冲击可能带来损失比较有限。从居民的心理预期角度来看,当灾区居民经历极端天气事件后,他们可能会重新评估灾害风险,发现实际的损失并没有达到预期的严重程度\citep{谢晓非2012心理台风眼效应研究综述,王大伟2014先前情绪和过度自信对灾难事件后继风险决策的影响},甚至没有达到免赔额。这种认知上的变化可能导致他们认为高额的保险投保并不必要,因此选择降低保险额度,以减少未来的保险费用支出。这种情况下,可以观察到灾区居民获得理赔概率相对于非灾区在灾后变化不大,因此居民保额需求减少。基于此本文定义是保单否理赔为$\text{Claimed}$,建立如下Logit DID回归模型:
\begin{equation}
    logit(\text{Claimed}_{it}) =\beta_0 +\beta_1 \text{Disaster}_i+\beta_2\text{Post}_{it}+\beta_3\text{Disaster}_i \times \text{Post}_{it} + Controls_{it}+ \varepsilon_{it}
    \label{eq:1}
\end{equation}

其次,居民可能低估极端降水冲击再次发生概率,“闪电不可能击中同一个地方两次”。居民可能认为,在极端降水冲击发生后,短期内再次发生极端降水冲击的概率较低\citep{szollosi2019simultaneous},因此平均保额有所下降,以减少不必要的保险费用。这种情况下,可以观察到居民续保的概率相对于非灾区在灾后变小,从而导致保险需求减少。基于此本文定义是保单否是否续保为$\text{Renew}$,建立如下Logit DID回归模型:
\begin{equation}
    logit(\text{Renew}_{it}) =\beta_0 +\beta_1 \text{Disaster}_i+\beta_2\text{Post}_{it}+\beta_3\text{Disaster}_i \times \text{Post}_{it} + Controls_{it}+ \varepsilon_{it}
    \label{eq:2}
\end{equation}

最后,平均保额的减少也可能是由于大量低保额的新保单加入。这种情况意味着灾区居民在受灾后风险感知增强,但新增的保险需求偏低,导致整体呈现出家财险平均保额减少。这种情况下,可以观察到灾区非续保保单的平均保额相对于灾区续保保单在灾后显著变小。基于此,本文定义非续保保单为$\text{First}$,针对灾区保单建立如下OLS DID回归模型:
\begin{equation}\label{eq:3}
    \log(\text{Coverage}_{it}) =\beta_0 + \beta_1 \text{First}_{it} + \beta_2 \text{Post}_{it}+\beta_3 \text{Post}_{it} \times \text{First}_{it} + Controls_{it}+ \varepsilon_{it}
\end{equation}

式(\ref{eq:1})、(\ref{eq:2})和(\ref{eq:3})的回归结果如表\ref{tab:renew}所示。从表中可以看出:
\begin{enumerate}
    \item 针对式(\ref{eq:1}),灾区居民在极端降水冲击后,$\text{Disaster}\times\text{Post}$显著为正,说明获得理赔概率相对于非灾区有所上升,说明保险起到了转移风险的作用。因此灾区冲击后平均保额下降原因并不是居民认为冲击后损失有限。
    \item 针对式(\ref{eq:2}),灾区居民在极端降水冲击后,$\text{Disaster}\times\text{Post}$显著为负,即灾后续保概率相对于非灾区显著下降,说明尽管降水冲击带来损失(式\ref{eq:1}),但居民认为短时间内再次发生极端降水冲击的概率较低,续保动力不足,平均保额下降。
    \item 针对式(\ref{eq:3}),$\text{First}$系数显著为负,灾区内首次投保的保单在冲击发生前平均保额就低于续保保单,与表\ref{tab:did1}与\ref{tab:did2}中非首次投保保额更高一致。但是交互项$\text{First}\times\text{Post}$系数并不显著,说明降水冲击对于首次投保保单的影响较小。
\end{enumerate}

\begin{table}[ht]
    \caption{进一步分析回归结果}\label{tab:renew}
    \centering
    
\begin{tabular}{@{\extracolsep{5pt}}lccc}
    \hline\hline
    \multicolumn{1}{c}{Dependent Variable}   & Claimed                                                              & Renew          & First            \\\cr\cline{2-4}
                         & (\ref{eq:1})                                                           & (\ref{eq:2})     & (\ref{eq:3})       \\
    \hline                                                                                                                          \\[-1.8ex]
    Disaster$\times$Post & 0.864$^{***}$                                                        & -2.055$^{***}$ &                  \\
                         & (0.275)                                                              & (0.294)        &                  \\
    First$\times$Post    &                                                                      &                & 0.082$^{}$       \\
                         &                                                                      &                & (0.076)          \\
    Disaster             & 0.457$^{*}$                                                          & -3.547$^{***}$ &                  \\
                         & (0.245)                                                              & (0.273)        &                  \\

    First                &                                                                      &                & -0.246$^{***}$   \\
                         &                                                                      &                & (0.066)          \\
    Intercept            & -16.703$^{}$                                                         & -25.564$^{}$   & 11.268$^{***}$   \\
                         & (2378.050)                                                           & (139.218)      & (0.118)          \\
    Post                 & -0.296$^{}$                                                          & 0.609$^{***}$  & -0.084$^{}$      \\
                         & (0.184)                                                              & (0.043)        & (0.076)          \\
    Controls             & Yes                                                                  & Yes            & Yes              \\
    \hline                                                                                                                          \\[-1.8ex]
    Observations         & 351719                                                               & 351719         & 44047            \\
    $R^2$                &                                                                      &                & 0.701            \\
    Adjusted $R^2$       &                                                                      &                & 0.701            \\
    Pseudo $R^2$         & 0.046                                                                & 0.402          &                  \\
    Residual Std. Error  &                                                                      &                & 0.436            \\
    F Statistic          &                                                                      &                & 5424.441$^{***}$ \\
    Year FE              & Yes                                                                  & Yes            & Yes              \\
    \hline
    \hline
    \textit{Note:}       & \multicolumn{3}{r}{$^{*}$p$<$0.1; $^{**}$p$<$0.05; $^{***}$p$<$0.01}                                     \\
\end{tabular}

\end{table}