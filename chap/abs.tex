%!TEX root = ../thesis.tex
\begin{cabstract}
    % 加入xx机制+准自然实验等名词
% 极端天气,降雨多发,以某公司家财险数据实证探究极端天气对家财险购买的影响
% 以什么为样本,用的什么方法
% 结果表明……(定量分析)
% 稳健性检验
% 进一步的分析揭示极端天气的过往经历、投保的金额等
% 本文的研究表明,为极端降雨与家财险购买的因果关系提供了经验证据,微观层面对个体风险决策的影响

    随着全球气候变化的加剧,极端天气事件对人类社会和经济活动的影响日益凸显。本研究旨在探讨极端天气冲击对家庭风险偏好的影响,特别是以家财险需求为例,分析极端天气事件如何影响家庭对财产保险的需求。通过实证分析极端降水对家财险需求的影响,本文深入探究了极端天气事件对家庭风险偏好的影响机制,填补了现有研究的空白。

    研究采用气象数据和承保理赔保险数据,运用回归分析和差分分析等方法,对家庭财产保险数据集进行分析。研究结果显示,极端天气事件的发生显著提高了居民的风险感知,从而促进了家财险需求的增加。具体而言,极端天气事件发生后,家财险的保额整体上显著提高。此外,临近灾区的地区由于风险感知增强,家财险需求显著增加,而直接受灾的地区由于对灾害的直观感受和对保险赔付能力的初步判断,家财险需求并未显著增加。

    本文的研究结果对于保险公司的产品开发、定价策略和风险管理具有重要指导意义,同时也为政府和监管机构提供了政策制定的依据。研究建议保险公司应通过产品设计与创新、动态定价和加强风险管理等措施,应对极端天气事件带来的挑战。政府则应通过税收优惠、补贴政策和营造公平透明的市场环境等措施,促进保险市场的健康发展,提高公众的保险保障能力。

\end{cabstract}
\begin{eabstract}
With the intensification of global climate change, the impact of extreme weather events on human society and economic activities has become increasingly prominent. This study aims to explore the impact of extreme weather shocks on household risk preferences, specifically using the demand for home property insurance as an example to analyze how extreme weather events affect families' demand for property insurance.

The study utilizes meteorological data from the China Meteorological Administration and insurance claims data from a property insurance company, employing DID methods to analyze the home property insurance dataset. The results show that the occurrence of extreme weather events significantly increases residents' risk perception, thereby promoting an increase in the demand for home property insurance. Specifically, after extreme weather events, the insurance coverage for home property increased significantly. Additionally, the demand for home property insurance in areas adjacent to disaster zones increased significantly due to heightened risk aversion, while in directly affected disaster areas, the demand for home property insurance did not increase significantly, possibly due to the direct experience of the disaster and preliminary judgments about the insurance payout capability.

The study suggests that insurance companies should respond to the challenges posed by extreme weather events through product innovation, dynamic pricing, and enhanced risk management measures. Governments should promote the healthy development of the insurance market and improve the insurance protection capacity of the public through tax incentives, subsidies, and the creation of a fair and transparent market environment.

\end{eabstract}
