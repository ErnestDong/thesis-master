%!TEX root = ../thesis.tex
\begin{cabstract}
    随着全球气候变化的加剧,极端天气事件对人类社会和经济活动的影响日益凸显。本研究基于极端降水冲击的准自然实验,通过构建双重差分模型,利用某财产保险公司1995年至2015年的家财险承保数据,评估极端降水事件冲击对家财险保额需求的影响。

    研究发现,极端天气冲击显著提高了近灾区居民对家财险的需求,平均保额在极端降水冲击后提升了约24\%。然而,对于直接受灾区域,家财险需求并未显著增加,反而平均保额可能减少18\%。回归结果通过了平行趋势检验和安慰剂检验,说明回归结果是稳健的。进一步的分析表明,灾区平均保额的降低与居民预期短期内再次受灾的概率较低有关,灾区受灾后损失较为严重,理赔大幅增加,但居民续保的概率大幅下降。

    此外,本研究还发现不同地区、不同时间节点以及不同价值的保险标的对象极端天气事件的反应存在异质性。西部地区居民在灾后对家财险的需求增加更为显著,可能与极端降水打破干旱预期有关。2008年金融危机和汶川地震之后,家庭对风险的感知显著增加,从而增加了对更高保额家财险的需求。保险标的价值越高,近灾区的保险需求增加越明显,而灾区的保险需求可能降低更明显。县区居民相较于地级市居民对极端天气事件的风险感知程度更高,保险需求增加的幅度相对较大。

    本研究丰富了气候变化社会经济影响的理解,也为制定科学的风险管理策略提供了一些理论依据。
\end{cabstract}
\begin{eabstract}
    As global climate change intensifies, the impact of extreme weather events on human social and economic activities has become increasingly prominent. This study is based on a quasi-natural experiment of extreme precipitation shocks, builds a double difference model, and uses the property insurance data of a property insurance company from 1995 to 2015 to assess the impact of extreme precipitation events on the demand for property insurance. The research found that extreme weather shocks have significantly increased the risk perception of residents in near-disaster areas, increasing the demand for family property insurance. The average insurance amount of residents in near-disaster areas increased by about 24\% after extreme rainfall shocks. However, the demand for home property insurance has not increased significantly for directly affected areas. On the contrary, the average insurance amount may decrease by 18\%, which may be related to residents' expectations of being affected again in the short term. In addition, this paper also found that there are heterogeneous reactions among insurance subject matter in different regions, different time points, and different values. This study enriches the understanding of the social and economic impacts of climate change and provides some theoretical basis for formulating scientific risk management strategies.
\end{eabstract}
