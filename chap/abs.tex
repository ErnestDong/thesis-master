%!TEX root = ../thesis.tex
\begin{cabstract}
    随着全球气候变化的加剧,极端天气事件对人类社会和经济活动的影响日益凸显。本研究基于极端降水冲击的准自然实验,以某公司家财险数据,采用差分分析方法探究极端天气冲击对家财险购买的影响。结果表明,极端天气冲击后,家财险的保额整体上显著提高7\%,20-50公里的近灾区相较于未受影响的地区保额进一步提升9\%,但是冲击20公里内的灾区由于受获得性启发机制及赔付的逆选择影响,保额下降了30\%,且回归结果通过了稳健性检验。进一步的分析揭示极端天气过往经历的适应性预期不足、巨灾导致的风险敏感度提升、保险标的的高价值以及城乡风险感知的差异都会对极端降水冲击带来的家财险需求有正向提升作用。本文的研究为极端降雨与家财险购买的因果关系提供了经验证据,在微观层面理解气候变化对个体风险决策的冲击提供了新的视角。
\end{cabstract}
\begin{eabstract}
    With the intensification of global climate change, the impact of extreme weather events on human social and economic activities has become increasingly prominent. Based on the quasi-natural experiment of extreme precipitation shock, this study uses the differential analysis method to explore the impact of extreme weather shock on the purchase of family property insurance based on the property insurance data of a company. The results show that after the extreme weather shock, the insurance amount of family property insurance is significantly increased by 7\%, and the insurance amount of the near disaster area of 20-50 km is further increased by 9\% compared with the unaffected areas. however, due to the influence of the adverse selection mechanism of acquisition inspiration and compensation, the insurance amount of the disaster area within 20 km decreased by 30\%, and the regression results passed the robustness test. Further analysis reveals that the lack of adaptive expectations of extreme weather past experience, the increase of risk sensitivity caused by catastrophes, the high value of the subject matter of insurance and the difference in risk perception between urban and rural areas will have a positive effect on the demand for family property insurance caused by extreme precipitation shocks. The study of this paper provides empirical evidence for the causal relationship between extreme rainfall and family property insurance purchase, and provides a new perspective to understand the impact of climate change on individual risk decision-making at the micro level.
\end{eabstract}
