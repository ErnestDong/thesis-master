\documentclass[a4paper,12pt]{ctexart}
\usepackage{amsmath}
\usepackage{amsfonts}
\usepackage{float}
\usepackage{enumerate}
\usepackage{svg}
\usepackage{graphicx}
\usepackage{booktabs}
\usepackage[hidelinks]{hyperref}
\setcounter{secnumdepth}{0}
\title{论文提纲}
\author{董晨阳}
\date{\today}
\usepackage[backend = biber, style = gb7714-2015ay, url=false,gbtitlelink=true]{biblatex}
\addbibresource{thesis.bib}
\begin{document}
\maketitle
% \clearpage
% \tableofcontents
% \listoffigures
\clearpage
\section{引言}
本文计划研究的核心问题是所在地方突然出现自然灾害后,家庭突击买家财险的激励有多大,以及和这种行为和购买企财险、健康险的行为对比。本文的贡献在于,通过整理出细粒度的脱敏保单数据,研究气候变化下对人行为偏好的影响,为保险公司提供更好的产品设计和营销策略,为政府提供更好的政策建议。

\section{文献综述}
已有文献研究主要集中在以下若干方面:

\begin{enumerate}
    \item 研究雾霾等气象变化对个人健康险的影响\citep{赵强2021空气污染对商业健康保险需求的影响——基于秦岭淮河地区的实证研究},基于模糊断点回归设计和中国北方集中供暖政策的准自然实验,估计了空气污染对中国居民商业健康保险需求的影响
    \item 研究台风等气象灾害对企业购买企财险行为的影响\citep{郑慧2016中国财险业台风灾害偿付能力评估——基于},研究表明中国财险业对台风灾害损失的实际赔付率仅为理论值的16\%,保险市场对台风灾害损失有较充裕的偿付空间。
    \item 定性研究地震等地质灾害对家庭购买家财险的影响\citep{于也雯0财产和生命双重风险约束下的家庭资产选择——基于地震风险的研究},研究表明地震保险的需求受到地震风险、保险产品、保险公司、政府政策、个人因素等因素的影响,发现地震的"财产风险效应"导致家庭的风险厌恶程度随地震震级的增加而提升,进而减少风险金融资产配置
\end{enumerate}

由于缺少细粒度的家财险数据,鲜有定量研究
\section{未来研究计划}
\subsection{变量选取与描述性统计}
TODO
\subsection{实证分析}
TODO
\subsection{异质性检验}
预期结论:气象平稳下保单数正常增长,并分析具体引致原因;突然出现暴雨后投保变多
\subsection{结论}
TODO
\printbibliography[heading = bibintoc]
\end{document}
