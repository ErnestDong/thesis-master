\documentclass[a4paper,12pt]{ctexbeamer}
\usetheme{CambridgeUS}
\usepackage{amsmath}
\usepackage{amsfonts}
\usepackage{float}
\usepackage{enumerate}
\usepackage{graphicx}
\usepackage{booktabs}
\usepackage{hyperref}
\usepackage[style = gb7714-2015ay, url=false,gbtitlelink=true]{biblatex}
\addbibresource{thesis.bib}
\title[论文答辩]{极端天气冲击对家财险需求的影响——以极端降水为例}
\author{董晨阳}
\date{\today}
\renewcommand{\bibfont}{\tiny}
\setbeamertemplate{bibliography item}{}
\begin{document}
\AtBeginSection[]
{
    \small
    \begin{frame}
        \frametitle{目录}
        \tableofcontents[
            sectionstyle=show/shaded,
            subsectionstyle=show/show/hide,
            subsubsectionstyle=show/show/show/hide
        ]
    \end{frame}
}
\maketitle


% ### 幻灯片 3: 研究背景与意义
% - 全球气候变化加剧,极端天气事件频发。
% - 探讨极端天气对家财险需求的影响,对理解气候变化的社会经济影响具有重要意义。

% ### 幻灯片 4: 研究目的
% - 评估极端降水事件对家财险保额需求的影响。
% - 丰富气候变化社会经济影响的理解。

% ### 幻灯片 5: 研究方法
% - 构建双重差分模型。
% - 利用1995-2015年的家财险数据进行分析。

% ### 幻灯片 6: 数据来源与样本选择
% - 数据来源:某财产保险公司家财险数据。
% - 样本选择:距离气象站20公里以内的标的样本。

% ### 幻灯片 7: 模型设定与变量定义
% - 介绍主要解释变量和控制变量。
% - 解释变量的经济含义及其预期作用。

% ### 幻灯片 8: 实证结果概览
% - 近灾区居民家财险需求增加约24%。
% - 直接受灾区家财险需求可能减少18%。

% ### 幻灯片 9: 稳健性检验
% - 平行趋势检验。
% - 安慰剂检验。
% - 其他稳健性检验。

% ### 幻灯片 10: 异质性分析
% - 地区异质性。
% - 时间异质性。
% - 保险标的价值异质性。
% - 城市级别异质性。

% ### 幻灯片 11: 进一步分析
% - 探讨灾区平均保额降低的可能原因。

% ### 幻灯片 12: 研究结论
% - 极端天气事件显著提高了居民的风险感知。
% - 不同地区居民对极端天气事件的反应存在差异。

% ### 幻灯片 13: 政策建议
% - 保险公司视角:创新产品与宣传,提高保障水平。
% - 政府视角:对保障不足的潜在灾区提供补贴。

% ### 幻灯片 14: 研究创新点
% - 以家庭财产保险购买为研究对象,讨论极端天气冲击的影响。
% - 探究极端天气事件冲击下,灾区和非灾区家庭财产保险需求的差异。

% ### 幻灯片 18: 问答环节
% - 准备回答评审团和听众的问题。

% 请注意,这只是一个大纲,具体的PPT制作还需要根据实际内容进行调整和详细设计。如果您需要更详细的内容或者有其他特殊要求,请告诉我。
\section{论文研究的目的和意义}
\begin{frame}
    \frametitle{<title>}


\end{frame}

\section{完成的工作和方法}
\section{获得的主要结论或提出的主要观点}
\section{论文的创新点}
\appendix
\nocite{*}
\begin{frame}[allowframebreaks]
    \frametitle{参考文献}
    \printbibliography
\end{frame}
\begin{frame}
    \centering
    \Huge Q\&A
\end{frame}
\end{document}
